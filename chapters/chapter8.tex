\chapter{Conclusions}
\label{chapter8}

\section{Limitations}

The end-result reached is certainly not perfect and has multiple limitations. One of these is the estimation latency brought by the RGB-depth synchronisation, which ultimately causes the chain to wait a couple of seconds before the next pose request is fired. This could limit the application usage of the package when high frame rates are required due to the dynamic environment.

Moreover, the ROS services used for the intra-module-communication, although offering an easy way to exchange messages between the various components, do limit the modularity of the package as all the modules would need to be run even if only the RGB detection is the only interest due to the client-server chained process.

Additionally, the conversion module does add to the project further computational costs, although these are contained. Nonetheless, solving the CvBridge compatibility issue with OpenCV 3.3.0 would allow the removal of an added step which could be easily integrated in the detection component.

Finally, both the detection and pose module have a degree of error. In fact, the detection process would certainly not be expected to perform as good in the detection of people as a human eye, and ultimately it could fail because of different environmental variables present or because it is something that the network has never seen before in the training data. Similarly, the 3D pose is most of the times incorrect up to a certain degree of error.

\section{Future Improvements}

To solve the latency issue only one options is really available. This consists in decreasing the publishing frequency of the raw RGB image such that it is synchronised with the raw depth-image stream more often. Increasing the publication stream of the depth-image is not an option as the depth-computation performed by the RGB-D sensor takes time.

The intra-module-communication limitation, differently from the latency issues, has an effective solution, which is to use just the publish-subscribe paradigm. Therefore getting rid of the synchronised chain process between the modules. Moreover, such an approach would allow the user to use only the desired module, although the pose component would still require the detection's publications.

Compiling from source the latest OpenCV release and resolving the issues with the already present static binding would make the conversion module redundant in its task, as the raw image conversion could be tackled using the updated CvBridge bindings directly in the detection module.

Bringing further improvements to the detection module would consist in potentially re-training the neural network with more data. Nonetheless, such an improvement is not really required given the already robust solution. However, to reduce the 3D error estimation a better calibration to the factory one can be carried out using available ROS packages, which could potentially improve the pixel projection and the 3D conversion result.

\section{Personal Reflection}

At the beginning of the academic year I didn't know much about Robotics nor Computer Vision and I ignored the fact the latter existed at all as a topic. Fast forward a couple of months into the year and I had the chance to discover some of the fundamental aspects behind both fields, through lectures and a group project where both aspects were put into practice. Towards the end of the module I knew that Robotics is what I wanted to pursue as a future career because of its challenges and the opportunity to work on so many different topics that intersects within it.

Hence the decision to carry out the following research project to further explore the field and advance in my personal knowledge. Throughout the project I learnt how to tackle a research task whose objectives are not precisely defined and how to implement abstract ideas read on papers in a constrained amount of time, increasing my organisational, analytical and problem solving skills.

I thoroughly enjoyed working on the project and I am happy to have created something useful for the open-source robotics community and for the University of Leeds RoboCup team, which I hope they will use to win many competitions in the future.

Moreover, thanks to this project I understood the importance that open-source projects have in such a field, and I am looking forward to contribute more to the open-source community in the future.





