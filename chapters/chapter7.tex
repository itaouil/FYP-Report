\chapter{Planning \& Project Management}
\label{chapter7}

In this chapter the planning and project management techniques used throughout the project in order to mitigate the inherent risk of the experimentation are discussed, with a focus on how the project was broke down into its phases, the source version control and other project management techniques. The Gantt chart with the project's weekly breakdowns is available in Appendix D.

\section{Change of aim}

The original aim of the project consisted in developing a software solution that would see the robot join people in ongoing discussions. Therefore both a perception and navigation component were present. However, the nature of the project and its experimentation approach, along with the intermediate report feedback and various discussions with the project's supervisor, where such that the former was changed so as not to incorporate the original human-robot behavioural aspect any more. Change of plans are normal in such projects and could possibly lead to a complete re-evaluation of the whole scope of the task including the previous background research. Although such re-evaluation happened to a certain extent, the majority of the research carried out was still valid for the revised aim, including the various techniques to perform person detection and distance estimation. Finally, the development process was not influenced by the change as this happened relatively early into the development stage.

\section{Planning}

As can be seen from the Gantt chart presented in Appendix D, the project was broken into three phases. The first phase consisted in the background research to explore the various ideas related to the project's aim and an overlapping first basic implementation of the detection module. The remaining stages consisted in developing a distance module to compute the stretch between the robot and the detection, and to ultimately extend the module with a 3D pose computation component to find the map location of the detection.

\subsection{Methodology}

For the software development stage a phase-driven agile type methodology was preferred over the classical waterfall approach for a couple of reasons. First and foremost the waterfall methodology is a linear one, meaning that the next stage of the development will not start as soon as the current one finishes in its entirety including testing and evaluation, thereby lacking flexibility in the development approach. Moreover, the waterfall method requires a good requirements analysis, which is not the case in this project given the exploratory nature.

Therefore the choice to use the iterative agile approach, where rapid deliveries of the components are emphasised. These functional backlog items are developed within sprints, which are time bounded period of times where the component needs to be completed. The agile approach, moreover, does not constraint the development of the functional components, allowing to work on several backlog items at the same time, something that happened few times during the project, when refinements were brought to previously developed features while developing the latest ones.

\section{Project Management}

\subsection{GitHub}

The project's code is hosted entirely on GitHub as a public repository\footnote{ \url{https://github.com/itaouil/human_position_estimation}}, which makes it easily accessible to everyone who wants to contribute to it or just make use of it. Moreover, GitHub offers other project management techniques that have been used throughout the development and which are presented below.

\subsection{Branching}

Git branching was an important aspect of the development, in order to separate the different phases in the development of the modules and the respective features. The branching approach enabled an overall safer experimenting process where already working features in the master branch were not affected by the ongoing development of others. 

Moreover, the approach taken in the development consisted of merging to master after every feature completion, followed by the deletion of the latter's local and remote branch, hence, none of the development branches are still available on GitHub anymore, nonetheless, 4 branches have been used during the implementation part of the project for the following features:

\begin{itemize}
  \item Person detection module
  \item OpenCV conversion module
  \item RGB to Depth mapping
  \item Depth to 3D pose
\end{itemize}

\subsection{Milestones \& Issues}

GitHub issues were also used to keep track of uncompleted tasks as well as bugs encountered during the different development phases. The standard GitHub issues labels have used to indicate the nature of the issue. Moreover, two Milestones have been identified, one for the marked presentation given before the Easter break and one for the final submission deadline. Issues have been assigned accordingly to their respective Milestone.

\subsection{License}

As previously stated the developed package developed will be made available to the open-source robotics community. Hence the need to make the repository publicly accessible, thereby granting permission to everyone to use and modify the software product, with the final aim of bringing further improvements to it.

Therefore, all the repository content, including source code and documentation was licensed under the \textbf{General Public License v3.0}.

\subsection{Wiki}

Documentation is really important for open-source projects to document the correct installation process, dependencies, usage and possible encountered issues. Therefore a detailed wiki was kept within GitHub which would serve as a reference for future usage.