\chapter{Introduction}
\label{chapter1}

\section{Context}

Rudimentary robots have been part of our lives since the early beginning of the industrial revolution. 

Nowadays we interact with servicing robots almost every day, starting from the toaster in the morning, TV and smartphones for personal entertainment to automated security checking at the airports we travel through. 

This is an unstoppable trend and in the near future robots will increasingly become more and more present in human environments in a variety of areas like assistance, grunt work, military, logistics and security.

Humans' interactions are defined by a set of precise protocols, in fact none us would carelessly jump in front of a queue or invade someone else's space in a common environment without asking permission.

Robots to be able to integrate in such an environment, and more importantly for humans to accept such an integration, will have to comply to some ethical, moral and social rules that govern our way of being humans.

In this project, the aim is to offer a starting point for such human-robot interaction, which is identifying people present in the surrounding with a certain level of confidence and robustness using several techniques and algorithms, with the final goal of creating an open-source tool to the Robotics community to tackle human aware navigation tasks.

From the intermediate report submitted at the beginning of the project some changes have occurred in the scope and planning of the project, these are outlined in appendix F. Finally, in appendix C access to the project repository, documentation and ROS package link can be found.

\section{Project Aim}

The aim of the project is to explore key-topics and main-ideas for human-aware navigation with focus on the perception side of things with the final goal of developing an open-source ROS package software product able to estimate human position in a dynamic and non-sparse environment through and ensemble approach of various algorithms and techniques, in order to facilitate human aware robot navigation.

Moreover, being this an open-source project offering interfaces for integration with other modules, modularity and ease of integration need to be features of this project.

\section{Problem Domain}

The main problems to be tackled are three, which consist in detecting the presence of people in the RGB image, get the human-robot distance and obtain the real pose (in terms of x, y and z) of the person or people in the map. 

All of these subtasks have challenges to them. In fact, the person detection module needs to be computationally efficient to be able to run real-time detections while being embedded in power constrained devices and keeping down the number of false positives and false negatives in the detection process.

The human-robot distance estimation needs to handle noise coming from the real world (lighting conditions, clutter and obstruction) while still being reliable as the real pose conversion heavily relies on a good depth estimation. 

Moreover, multiple sensors will be used for the development of the project, which on one side offers a more robust estimations process as the different sensors can complement the weaknesses of one another, however this increases the level of complexity in the integration and the possible accessibility to the ROS package, although the sensors used should be available on all standard robotics systems.

\section{Project Objectives}

The following objectives for the project are defined:

\begin{enumerate}
  \item Obtain a level of proficiency in using the ROS\footnote{Robot Operating
System} platform and its tools, including RVIZ, custom messages and packaging using CMAKE.
	\item Obtain a level of proficiency in using Robotics sensors such as RGB-D and laser-scans and in integrating them.
  \item Explore available person detection computer vision algorithms and the state-of-the-art techniques for the task.
  \item Explore available distance estimation techniques using RGB-D sensors and plain RGB data.
\end{enumerate}

\section{Project Deliverables}

The following deliverables are to be expected:

\begin{enumerate}
  \item Develop a human detection module using the robot's built-in RGB sensor.
  \item Develop a human-robot depth module to estimate the distance between the robot and the detection/s using the robot's built-in RGB-D sensor or other computer vision techniques.
  \item Integrate a leg detector algorithm for a more robust pose estimation outside of the RGB-D range.
  \item Create visual markers on RVIZ to show people's position in the map.
  \item Release the package ROS.
\end{enumerate}