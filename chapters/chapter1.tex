\chapter{Introduction}
\label{chapter1}

\section{Context}

Rudimentary robots have been part of our lives since the early beginning of the industrial revolution. 

Nowadays we interact with servicing robots almost every day, starting from the toaster in the morning, to the TV and smart-phone for personal entertainment to automated security checking at the airports we travel through. 

This is an unstoppable trend and in the near future robots will increasingly become more and more present in human environments in a variety of areas like assistance, grunt work, military, logistics and security.

Humans' interactions are defined by a set of precise protocols, in fact none us would carelessly jump in front of a queue or invade someone else's space in a common environment without asking permission.

Robots to be able to integrate in such an environment, and more importantly for humans to accept such an integration, will have to comply to some ethical, moral and social rules that govern our way of being humans.

\section{Aim}

The aim of the project is to explore key-topics and main-ideas for human-aware navigation and human-robot interaction with the final goal of developing a ROS package software product able to estimate human position in a dynamic and non-sparse environment through and ensemble approach of various algorithms and techniques, in order to facilitate robotics human awareness.

\section{Domain}

The problems to be tackled for the accomplishment of the goal are many, including being able to detect human presence with good real-time performance and accuracy in spaces where the robot will not have a clear view of the person or when the person is not clearly standing in front of the camera, therefore avoiding to have as many false positives or false negatives instances as possible. Moreover, the final system has to be able to estimate the position of the person or people in the space, therefore adding depth and out of sensory range challenges which will need to dealt with.

\section{Objectives}

The following objectives for the project are defined:

\begin{enumerate}
  \item Explore with common and state-of-the-art computer vision algorithms for object detection tasks, focusing principally on human detection.
  \item Obtain a level of proficiency in using the ROS platform and its tools.
  \item Develop a human detection module using the robot's built-in RGB sensor.
  \item Develop a leg detection module for person pose estimation and long distance detection using the robot's built-in laser-scan.
  \item Develop a depth estimation module to estimate robot-detections distance using the robot's built-in RGB and RGB-D sensor or other computer vision techniques.
  \item Integrate the developed modules and create a ROS package able to estimate the position of the detected persons in the map, offering RVIZ marker visualisation of the detections in the map as well as custom detection messages with information about the detections, such as position in the map, confidence of the detection, bounding box centre, etc.
\end{enumerate}